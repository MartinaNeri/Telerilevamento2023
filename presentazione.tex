\documentclass{beamer}
\usepackage{graphicx}
\usepackage{hyperref}
\usepackage{caption}
\usepackage{amsmath}
\usepackage{booktabs}
\usepackage{listings}
\usepackage{xcolor}

\setlength{\abovecaptionskip}{5pt}

\usetheme{metropolis}

% Personalizzazione dei colori
\definecolor{verdepetrolio}{RGB}{50, 130, 102}
\definecolor{bianco}{RGB}{255, 255, 255}
\setbeamercolor{frametitle}{bg=verdepetrolio}
\setbeamercolor{alerted text}{fg=bianco}

% Imposta il font predefinito su Helvetica
\renewcommand{\familydefault}{\sfdefault}

% Stile per il codice
\lstset{
    basicstyle=\ttfamily\footnotesize,
    keywordstyle=\color{verdepetrolio}\bfseries,
    commentstyle=\color{gray},
    stringstyle=\color{orange},
    frame=single,
    breaklines=true,
    columns=fullflexible,
    language=R
}

\title{\textbf{Analisi di Telerilevamento con R}}
\subtitle{Analisi temporale dei Carpazi rumeni tra il 2015 e il 2024}
\author{Martina Neri}
%\date{}

\begin{document}

\maketitle

\begin{frame}
\frametitle{Outline}
\tableofcontents
\end{frame}

\section{Introduzione}

\begin{frame}{\textbf{Scopo dello studio}}
Confrontare immagini satellitari tra il 2015 e il 2024 per:
\begin{itemize}
    \item Analizzare variazioni di vegetazione tramite l'indice \textbf{NDVI}.
    \item Identificare cambiamenti significativi nella copertura vegetale.
    \item Eseguire una \textbf{Principal Component Analysis} (PCA) per evidenziare aree vulnerabili.
\end{itemize}
\end{frame}

\section{Materiali e Metodi}

\begin{frame}{\textbf{Dati e Software}}
\begin{itemize}
    \item \textbf{Dati}: Immagini Landsat 8 degli anni 2015 e 2024.
    \item \textbf{Software}: Analisi effettuata con il linguaggio R e pacchetti come:
    \begin{itemize}
        \item \texttt{raster}
        \item \texttt{ggplot2}
        \item \texttt{RStoolbox}
    \end{itemize}
\end{itemize}
\end{frame}

\begin{frame}[fragile]{\textbf{Script di importazione ed elaborazione dei dati}}
\begin{lstlisting}
# Caricamento delle immagini satellitari
rlist_2015 <- list.files(pattern = "Landsat8_2015")
import_2015 <- lapply(rlist_2015, raster)
img_2015 <- stack(import_2015)

rlist_2024 <- list.files(pattern = "Landsat8_2024_LowCloud")
import_2024 <- lapply(rlist_2024, raster)
img_2024 <- stack(import_2024)

# Visualizzazione delle immagini
par(mfrow = c(1, 2))
plotRGB(img_2015, 3, 2, 1, stretch = "lin")
plotRGB(img_2024, 3, 2, 1, stretch = "lin")
\end{lstlisting}
\end{frame}

\section{Risultati}

\subsection{Calcolo e confronto degli indici spettrali}

\begin{frame}{\textbf{NDVI: Calcolo e Confronto}}
\textbf{Formula NDVI}:
\[
NDVI = \frac{\text{NIR} - \text{Rosso}}{\text{NIR} + \text{Rosso}}
\]
\begin{itemize}
    \item Calcolo del \textbf{NDVI} per entrambi gli anni.
    \item Differenza tra il NDVI 2024 e il NDVI 2015 per evidenziare variazioni significative.
\end{itemize}
\end{frame}

\begin{frame}[fragile]{\textbf{Calcolo in R}}
\begin{lstlisting}
# Calcolo del NDVI
NDVI_2015 <- (img_2015[[4]] - img_2015[[3]]) / (img_2015[[4]] + img_2015[[3]])
NDVI_2024 <- (img_2024[[4]] - img_2024[[3]]) / (img_2024[[4]] + img_2024[[3]])
NDVI_dif <- NDVI_2024 - NDVI_2015

# Visualizzazione
par(mfrow = c(1, 3))
plot(NDVI_2015, main = "NDVI 2015")
plot(NDVI_2024, main = "NDVI 2024")
plot(NDVI_dif, main = "Differenza NDVI")
\end{lstlisting}
\end{frame}

\subsection{Test statistici e PCA}

\begin{frame}[fragile]{\textbf{Test statistico sul NDVI}}
\begin{lstlisting}
# Test t accoppiato tra NDVI 2015 e NDVI 2024
ndvi_values_2015 <- getValues(NDVI_2015)
ndvi_values_2024 <- getValues(NDVI_2024)
t_test_result <- t.test(ndvi_values_2015, ndvi_values_2024, paired = TRUE)
print(t_test_result)
\end{lstlisting}
\textbf{Risultati:}
\begin{itemize}
    \item \textbf{t = -107.12}, \textbf{p-value < 2.2e-16}.
    \item Differenza significativa tra NDVI 2015 e 2024.
\end{itemize}
\end{frame}

\begin{frame}[fragile]{\textbf{PCA sulle immagini NDVI}}
\begin{lstlisting}
# PCA su NDVI
stack_ndvi <- stack(NDVI_2015, NDVI_2024)
PCA_ndvi <- prcomp(sampleRandom(stack_ndvi, 10000, as.data.frame = TRUE))
PCI_raster <- predict(stack_ndvi, PCA_ndvi, index = 1)

# Visualizzazione della prima componente principale
plot(PCI_raster, main = "Prima Componente Principale (PC1)")
\end{lstlisting}
\end{frame}

\section{Conclusioni}

\begin{frame}{\textbf{Conclusioni}}
\begin{itemize}
    \item Differenze significative nel \textbf{NDVI} suggeriscono una riduzione della vegetazione.
    \item La \textbf{PCA} evidenzia zone con maggiori variazioni ambientali.
    \item L'analisi con R si è rivelata un valido strumento per il monitoraggio dei cambiamenti ambientali.
\end{itemize}
\end{frame}

\section{Analisi future}

\begin{frame}{\textbf{Analisi future}}
\begin{itemize}
    \item Sarebbe interessante fare un analisi della differenza dello stato di salute delle foreste nei carpazi al confine tra Romania e Ucraina??????????????????
  
\end{itemize}
\end{frame}


\begin{frame}{}
\centering
\textbf{Grazie per l'attenzione!}
\end{frame}

\end{document}
