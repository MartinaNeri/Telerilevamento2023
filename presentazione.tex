\documentclass{beamer}
\usepackage{graphicx} % Required for inserting images

\title{Presentazione_esame_telerilevamento}
\author{martina neri}
\date{November 2024}

\begin{document}


\section{Introduction}

\documentclass{beamer}
\usepackage{graphicx}
\usepackage{hyperref}
\usepackage{caption}

\setlength{\abovecaptionskip}{5pt}

\usetheme{metropolis}

% Personalizzazione dei colori
\definecolor{verdescuro}{RGB}{34, 87, 45}
\definecolor{bianco}{RGB}{255, 255, 255}
\setbeamercolor{frametitle}{bg=verdescuro}
\setbeamercolor{alerted text}{fg=bianco}

% Imposta il font predefinito su Helvetica
\renewcommand{\familydefault}{\sfdefault}

\title{\textbf{Analisi della Deforestazione in Romania}}
\subtitle{Studio mediante Telerilevamento 2017-2024}
\author{Analisi Satellitare}
%\date{}

\begin{document}
\maketitle

\begin{frame}
\frametitle{Outline}
\tableofcontents
\end{frame}

% Sezione 1: introduzione
\section{Introduzione}

\begin{frame}{\textbf{Scopo dello studio}}
Monitoraggio della deforestazione nell'area di Bălan, Romania, attraverso l'analisi di immagini satellitari \textbf{Sentinel-2} nel periodo 2017-2024
\end{frame}

\begin{frame}{\textbf{Obiettivi}}
\begin{itemize}
    \item Acquisizione e analisi di immagini \textbf{Sentinel-2} (2017-2024)
    \item \pause Calcolo di \textbf{indici spettrali} tramite R
    \item \pause Analisi delle variazioni della copertura forestale
    \item \pause Quantificazione della perdita di vegetazione
\end{itemize}
\end{frame}

% Sezione 2: materiali e metodi
\section{Materiali e Metodi}

\begin{frame}{\textbf{Materiali e Metodi}}
\begin{itemize}
    \item Download e preprocessing delle immagini \textbf{Sentinel-2}
    \item \pause Calcolo degli \textbf{indici spettrali}:
    \begin{itemize}
        \item DVI (Difference Vegetation Index)
        \item NDVI (Normalized Difference Vegetation Index)
        \item EVI (Enhanced Vegetation Index)
    \end{itemize}
    \item \pause Analisi \textbf{PCA} (Principal Component Analysis)
    \item \pause Analisi statistica (\textbf{T-test})
    \item \pause Classificazione della \textbf{Land Cover}
\end{itemize}
\end{frame}

\section{Risultati}

\subsection{Analisi degli Indici Spettrali}

\begin{frame}{\textbf{NDVI: Normalized Difference Vegetation Index}}
\begin{itemize}
    \item Formula: $NDVI = \frac{NIR - Red}{NIR + Red}$
    \item \pause Varia da -1 a +1
    \item \pause Valori più alti indicano vegetazione più sana
    \item \pause Perdita significativa rilevata:
    \begin{itemize}
        \item Numero di pixel con perdita: \textbf{87,475}
        \item Percentuale area interessata: \textbf{2.08\%}
    \end{itemize}
\end{itemize}
\end{frame}

\begin{frame}{\textbf{Enhanced Vegetation Index (EVI)}}
\begin{equation}
EVI = G \times \frac{NIR - Red}{NIR + C1 \times Red - C2 \times Blue + L}
\end{equation}
Dove:
\begin{itemize}
    \item G = 2.5 (fattore di guadagno)
    \item C1 = 6
    \item C2 = 7.5
    \item L = 1 (fattore di correzione del suolo)
\end{itemize}
\end{frame}

\subsection{Analisi PCA}

\begin{frame}{\textbf{Principal Component Analysis}}
\begin{itemize}
    \item Riduzione della dimensionalità dei dati
    \item \pause Identificazione delle aree con maggior variazione
    \item \pause Risultati:
    \begin{itemize}
        \item PC1: evidenzia le zone di maggior cambiamento
        \item \pause Campionamento casuale di 10,000 punti
        \item \pause Validazione statistica tramite T-test
    \end{itemize}
\end{itemize}
\end{frame}

\subsection{Land Cover Analysis}

\begin{frame}{\textbf{Analisi della Copertura del Suolo}}
Classificazione binaria:
\begin{itemize}
    \item Copertura vegetale \textbf{buona}
    \item Copertura vegetale \textbf{ridotta/assente}
\end{itemize}
\pause
Metodologia:
\begin{itemize}
    \item Clustering K-means (k=2)
    \item \pause Calcolo delle frequenze e percentuali
    \item \pause Confronto temporale 2017-2024
\end{itemize}
\end{frame}

\section{Conclusioni}

\begin{frame}{\textbf{Conclusioni}}
\begin{itemize}
    \item Perdita significativa di copertura forestale (\textbf{2.08\%} dell'area)
    \item \pause L'analisi PCA ha evidenziato i pattern di deforestazione
    \item \pause Il T-test conferma la significatività statistica dei cambiamenti
    \item \pause Necessità di monitoraggio continuo e misure di conservazione
\end{itemize}
\end{frame}

\begin{frame}{\textbf{Prospettive Future}}
\begin{itemize}
    \item Estensione dell'analisi ad altre aree della Romania
    \item \pause Integrazione con dati socio-economici
    \item \pause Sviluppo di un sistema di early warning
    \item \pause Supporto alle politiche di conservazione forestale
\end{itemize}
\end{frame}

\begin{frame}{}
\centering
\textbf{Grazie per l'attenzione!}
\end{frame}

\end{document}

\end{document}


