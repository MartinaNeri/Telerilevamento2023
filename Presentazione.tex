\documentclass{beamer}
\usepackage{graphicx}
\usepackage{hyperref}
\usepackage{caption}
\usepackage{amsmath}
\usepackage{listings}
\usepackage{xcolor}

\setlength{\abovecaptionskip}{5pt}
\usetheme{metropolis}

% Personalizzazione dei colori
\definecolor{verdeScuro}{RGB}{0, 100, 0}
\setbeamercolor{frametitle}{bg=verdeScuro, fg=white}
\setbeamercolor{title}{bg=verdeScuro, fg=verdeScuro}
\setbeamercolor{footline}{bg=verdeScuro, fg=white}
\setbeamercolor{background canvas}{bg=white}
\setbeamercolor{title in head/foot}{fg=white, bg=verdeScuro}

% Imposta il font predefinito su Helvetica
\renewcommand{\familydefault}{\sfdefault}

% Stile per il codice
\lstset{
    basicstyle=\ttfamily\footnotesize,
    keywordstyle=\color{verdeScuro}\bfseries,
    commentstyle=\color{gray},
    stringstyle=\color{red},
    frame=single,
    breaklines=true,
    columns=fullflexible,
    language=R
}

\title{\textbf{\textcolor{verdeScuro}{Analisi Incendi Australia 2019-2025}}}
\subtitle{Monitoraggio della foresta australiana colpita dagli incendi tra il 2019 e il 2025}
\author{Martina Neri}
%\date{}

\begin{document}

\maketitle

\begin{frame}
\frametitle{Outline}
\tableofcontents
\end{frame}

\section{Introduzione}

\begin{frame}{\textbf{Scopo dello studio}}
Confrontare immagini satellitari tra il 2019 e il 2025 per:
\begin{itemize}
    \item Analizzare variazioni di vegetazione tramite l'indice \textbf{NDVI}.
    \item Identificare cambiamenti significativi nella copertura vegetale.
    \item Eseguire una \textbf{Principal Component Analysis} (PCA) per evidenziare aree vulnerabili.
\end{itemize}
\end{frame}

\section{Materiali e Metodi}

\begin{frame}{\textbf{Dati e Software}}
\begin{itemize}
    \item \textbf{Dati}: Immagini satellitari degli anni 2019, 2020 e 2025.
    \item \textbf{Software}: Analisi effettuata con il linguaggio R e pacchetti come:
    \begin{itemize}
        \item \texttt{raster}
        \item \texttt{ggplot2}
        \item \texttt{viridis}
    \end{itemize}
\end{itemize}
\end{frame}

\begin{frame}[fragile]{\textbf{Caricamento delle immagini satellitari}}
\begin{lstlisting}
# Caricamento delle immagini satellitari
rlist_2019 <- list.files(pattern = "2019-01-16")
import_2019 <- lapply(rlist_2019, raster)
img_2019 <- stack(import_2019)

rlist_2020 <- list.files(pattern = "2020-01-21")
import_2020 <- lapply(rlist_2020, raster)
img_2020 <- stack(import_2020)

rlist_2025 <- list.files(pattern = "2025-01-14")
import_2025 <- lapply(rlist_2025, raster)
img_2025 <- stack(import_2025)

# Visualizzazione delle immagini
par(mfrow = c(1, 3))
plotRGB(img_2019, 3, 2, 1, stretch = "lin")
plotRGB(img_2020, 3, 2, 1, stretch = "lin")
plotRGB(img_2025, 3, 2, 1, stretch = "lin")
\end{lstlisting}
\end{frame}

\section{Risultati}

\subsection{Calcolo e confronto degli indici spettrali}

\begin{frame}{\textbf{NDVI: Calcolo e Confronto}}
\textbf{Formula NDVI}:
\[
NDVI = \frac{\text{NIR} - \text{Rosso}}{\text{NIR} + \text{Rosso}}
\]
\begin{itemize}
    \item Calcolo del \textbf{NDVI} per gli anni 2019, 2020 e 2025.
    \item Differenza tra il NDVI 2020 e il NDVI 2019, e tra il NDVI 2025 e il NDVI 2020, per evidenziare variazioni significative.
\end{itemize}
\end{frame}

\begin{frame}[fragile]{\textbf{Calcolo in R}}
\begin{lstlisting}
# Calcolo del NDVI
NDVI_2019 <- (img_2019[[4]] - img_2019[[3]]) / (img_2019[[4]] + img_2019[[3]])
NDVI_2020 <- (img_2020[[4]] - img_2020[[3]]) / (img_2020[[4]] + img_2020[[3]])
NDVI_2025 <- (img_2025[[4]] - img_2025[[3]]) / (img_2025[[4]] + img_2025[[3]])
NDVI_diff_2020 <- NDVI_2020 - NDVI_2019
NDVI_diff_2025 <- NDVI_2025 - NDVI_2020

# Visualizzazione
par(mfrow = c(1, 3))
plot(NDVI_2019, main = "NDVI 2019")
plot(NDVI_2020, main = "NDVI 2020")
plot(NDVI_2025, main = "NDVI 2025")

par(mfrow = c(1, 2))
plot(NDVI_diff_2020, main = "Differenza NDVI 2020-2019")
plot(NDVI_diff_2025, main = "Differenza NDVI 2025-2020")
\end{lstlisting}
\end{frame}

\subsection{Test statistici e PCA}

\begin{frame}[fragile]{\textbf{Test statistico sul NDVI}}
\begin{lstlisting}
# Paired t-Test tra NDVI 2019 e 2020, e tra NDVI 2019 e 2025
ndvi_values_2019 <- getValues(NDVI_2019)
ndvi_values_2020 <- getValues(NDVI_2020)
ndvi_values_2025 <- getValues(NDVI_2025)
t_test_2020 <- t.test(ndvi_values_2019, ndvi_values_2020, paired = TRUE)
t_test_2025 <- t.test(ndvi_values_2019, ndvi_values_2025, paired = TRUE)
print(t_test_2020)
print(t_test_2025)
\end{lstlisting}
\textbf{Risultati:}
\begin{itemize}
    \item \textbf{t = -107.12}, \textbf{p-value < 2.2e-16} per il confronto tra 2019 e 2020.
    \item \textbf{t = -112.45}, \textbf{p-value < 2.2e-16} per il confronto tra 2019 e 2025.
\end{itemize}
\end{frame}

\begin{frame}[fragile]{\textbf{PCA sulle immagini NDVI}}
\begin{lstlisting}
# PCA su NDVI
stack_ndvi <- stack(NDVI_2019, NDVI_2020, NDVI_2025)
PCA_ndvi <- prcomp(sampleRandom(stack_ndvi, 10000, as.data.frame = TRUE))
PCI_raster <- predict(stack_ndvi, PCA_ndvi, index = 1)

# Visualizzazione della prima componente principale
plot(PCI_raster, main = "Prima Componente Principale (PC1)")
\end{lstlisting}
\end{frame}

\section{Conclusioni}

\begin{frame}{\textbf{Conclusioni}}
\begin{itemize}
    \item Differenze significative nel \textbf{NDVI} suggeriscono una riduzione della vegetazione immediatamente dopo l'incendio nel 2020.
    \item La \textbf{PCA} evidenzia zone con maggiori variazioni ambientali.
    \item L'analisi con R si è rivelata un valido strumento per il monitoraggio dei cambiamenti ambientali.
\end{itemize}
\end{frame}

\section{Analisi future}

\begin{frame}{\textbf{Analisi future}}
\begin{itemize}
    \item Estendere l'analisi ad altre regioni colpite dagli incendi in Australia.
    \item Confrontare i risultati con dati meteorologici per determinare l'influenza del clima sulla rigenerazione della vegetazione.
    \item Integrare dati di altre fonti satellitari per un monitoraggio più completo.
\end{itemize}
\end{frame}

\begin{frame}{}
\centering
\textbf{Grazie per l'attenzione!}
\end{frame}

\end{document}
