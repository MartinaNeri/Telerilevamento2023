\documentclass{beamer}
\usepackage{graphicx}
\usepackage{hyperref}
\usepackage{caption}
\usepackage{amsmath}
\usepackage{listings}
\usepackage{xcolor}

\setlength{\abovecaptionskip}{5pt}
\usetheme{metropolis}

% Personalizzazione dei colori
\definecolor{verdeScuro}{RGB}{0, 100, 0}
\setbeamercolor{frametitle}{bg=verdeScuro, fg=white}
\setbeamercolor{title}{bg=verdeScuro, fg=verdeScuro}
\setbeamercolor{footline}{bg=verdeScuro, fg=white}
\setbeamercolor{background canvas}{bg=white}
\setbeamercolor{title in head/foot}{fg=white, bg=verdeScuro}

% Imposta il font predefinito su Helvetica
\renewcommand{\familydefault}{\sfdefault}

% Stile per il codice
\lstset{
    basicstyle=\ttfamily\footnotesize,
    keywordstyle=\color{verdeScuro}\bfseries,
    commentstyle=\color{gray},
    stringstyle=\color{red},
    frame=single,
    breaklines=true,
    columns=fullflexible,
    language=R
}

\title{\textbf{\textcolor{verdeScuro}{Analisi Incendi Australia 2019-2025}}}
\subtitle{Monitoraggio della foresta australiana colpita dagli incendi tra il 2019 e il 2025}
\author{Martina Neri}
%\date{}

\begin{document}

\maketitle

\begin{frame}
\frametitle{Outline}
\tableofcontents
\end{frame}

\section{Introduzione}

\begin{frame}{\textbf{Introduzione}}
\textbf{Titolo:} Analisi Incendi in Australia (2019-2020-2025)
\newline
\newline
\textbf{Obiettivo:} Esaminare gli effetti degli incendi forestali del 2019 in Australia sull'ambiente tramite immagini satellitari.
\newline
\newline
\textbf{Approccio:} Confronto tra immagini pre e post incendio e analisi temporale degli indici spettrali e della Land cover.
\end{frame}

\section{Dati e Software Utilizzati}

\begin{frame}{\textbf{Dati e Software Utilizzati}}
\textbf{Dataset:} Immagini satellitari di tre date chiave (2019, 2020, 2025).
\newline
\newline
\textbf{Software:} R con i pacchetti \texttt{raster}, \texttt{ggplot2}, \texttt{viridis}.
\newline
\newline
\textbf{Indici Analizzati:}
\begin{itemize}
    \item DVI (Difference Vegetation Index)
    \item NDVI (Normalized Difference Vegetation Index)
\end{itemize}
\newline
\textbf{Strumenti di Analisi:}
\begin{itemize}
    \item PCA (Analisi delle Componenti Principali)
    \item Classificazione Land Cover (K-Means)
\end{itemize}
\end{frame}

\section{Caricamento e Visualizzazione dei Dati}

\begin{frame}[fragile]{\textbf{Caricamento e Visualizzazione dei Dati}}
\textbf{Impostazione della working directory:}
\begin{lstlisting}
setwd("C:/lab/Esame_telerilevamento/prova")
\end{lstlisting}

\textbf{Importazione delle immagini:}
\begin{lstlisting}
rlist_2019 <- list.files(pattern = "2019-01-26")
import_2019 <- lapply(rlist_2019, raster)
img_2019 <- stack(import_2019)
#stesso procedimento per 2020 e 2025
\end{lstlisting}

\textbf{Visualizzazione delle immagini:}
\begin{lstlisting}
par(mfrow = c(1, 2))
plotRGB(img_2019, 3, 2, 1, stretch = "lin") #True color
plotRGB(img_2020, 4, 3, 2, stretch = "lin") #NIR
dev.off()
#stesso procedimento per 2020 e 2025
\end{lstlisting}
\end{frame}

\section{Analisi degli Indici Spettrali}

\begin{frame}[fragile]{\textbf{Analisi degli Indici Spettrali}}
\textbf{DVI (Difference Vegetation Index):}
\begin{lstlisting}
# Formula
DVI_2019 <- img_2019[[4]] - img_2019[[3]]
#stesso procedimento per 2020 e 2025

# Differenze tra anni
DVI_diff_2020 <- DVI_2020 - DVI_2019 #danni dati dall'incendio
DVI_diff_2025 <- DVI_2025 - DVI_2020 #ripristino vegetazione

# Visualizzazione
par(mfrow = c(1, 3))
plot(DVI_2019, col = viridis(10), main = "DVI 2019")
plot(DVI_2020, col = viridis(10), main = "DVI 2020")
plot(DVI_2025, col = viridis(10), main = "DVI 2025")
dev.off()

\end{lstlisting}
\end{frame}

\begin{frame}{\textbf{Immagini DVI}}
\begin{columns}
    \column{0.33\textwidth}
    \centering
    \textbf{DVI 2019}
    \includegraphics[width=\textwidth]{dvi_2019.png}
    
    \column{0.33\textwidth}
    \centering
    \textbf{DVI 2020}
    \includegraphics[width=\textwidth]{dvi_2020.png}
    
    \column{0.33\textwidth}
    \centering
    \textbf{DVI 2025}
    \includegraphics[width=\textwidth]{dvi_2025.png}
\end{columns}
\end{frame}

\begin{frame}[fragile]{\textbf{Indice NDVI}}
\textbf{NDVI (Normalized Difference Vegetation Index):}
\begin{lstlisting}
# Formula
NDVI_2019 <- DVI_2019/ (img_2019[[4]] + img_2019[[3]])
#stesso procedimento per 2020 e 2025

# Cambiamenti temporali
NDVI_diff_2020 <- NDVI_2020 - NDVI_2019 #danni dati dall'incendio
NDVI_diff_2025 <- NDVI_2025 - NDVI_2020 #ripristino vegetazione

# Visualizzazione
par(mfrow = c(1, 3))
plot(NDVI_2019, col = viridis(10), main = "NDVI 2019")
plot(NDVI_2020, col = viridis(10), main = "NDVI 2020")
plot(NDVI_2025, col = viridis(10), main = "NDVI 2025")
dev.off()
\end{lstlisting}
\end{frame}

\begin{frame}{\textbf{Immagini NDVI}}
\begin{columns}
    \column{0.33\textwidth}
    \centering
    \textbf{NDVI 2019}
    \includegraphics[width=\textwidth]{ndvi_2019.png}
    
    \column{0.33\textwidth}
    \centering
    \textbf{NDVI 2020}
    \includegraphics[width=\textwidth]{ndvi_2020.png}
    
    \column{0.33\textwidth}
    \centering
    \textbf{NDVI 2025}
    \includegraphics[width=\textwidth]{ndvi_2025.png}
\end{columns}
\end{frame}


\begin{frame}[fragile]{\textbf{Indice NDVI}}
\textbf{NDVI (Normalized Difference Vegetation Index):}
\begin{lstlisting}
# Visualizzazione delle differenze
par(mfrow = c(1, 2))
plot(NDVI_diff_2020, col = viridis(10), main = "Differenza NDVI 2020-2019")
plot(NDVI_diff_2025, col = viridis(10), main = "Differenza NDVI 2025-2020")
dev.off()
\end{lstlisting}
\end{frame}

\begin{frame}{\textbf{Differenze NDVI}}
\begin{columns}
    \column{0.5\textwidth}
    \centering
    \textbf{NDVI 2019-2020}
    \includegraphics[width=\textwidth]{NDVI 2019-2020.png}
    
    \column{0.5\textwidth}
    \centering
    \textbf{NDVI 2020-2025}
    \includegraphics[width=\textwidth]{NDVI 2020-2025.png}
\end{columns}
\end{frame}


\section{PCA (Analisi delle Componenti Principali)}

\begin{frame}[fragile]{\textbf{PCA (Analisi delle Componenti Principali)}}
\textbf{Obiettivo:} Identificare pattern principali nei valori NDVI.
\newline
\newline
\textbf{Procedura:}
\begin{lstlisting}
# Campionamento casuale
NDVI_stack <- stack(NDVI_2019, NDVI_2020, NDVI_2025)
sample_pixels <- sampleRandom(NDVI_stack, size = 10000)

# Esecuzione PCA
PCA_ndvi <- prcomp(sample_pixels, scale. = TRUE)

# Proiezione
PCI_raster <- predict(NDVI_stack, PCA_ndvi, index = 1)
plot(PCI_raster, main = "Prima Componente Principale (PC1)", col = viridis(100))
\end{lstlisting}
\end{frame}

\begin{frame}{\textbf{Immagini PCA}}
\begin{columns}
    \column{0.5\textwidth}
    \centering
    \textbf{PCA Componente 1}
    \includegraphics[width=\textwidth]{pca_1.png}
    
    \column{0.5\textwidth}
    \centering
    \textbf{Summary della PCA}
    \begin{table}
        \centering
        \begin{tabular}{lccc}
            \toprule
            & PC1 \\
            \midrule
            Standard deviation     & 1.4712\\
            Proportion of Variance & 0.7215\\
            Cumulative Proportion  & 0.7215\\
            \bottomrule
        \end{tabular}
    \end{table}
\end{columns}
\end{frame}

\section{Classificazione Land Cover}

\begin{frame}[fragile]{\textbf{Classificazione Land Cover}}
\textbf{Metodo:} K-Means clustering su immagini raster.
\newline
\newline
\textbf{Risultati per il 2019:}
\begin{lstlisting}
# Estrazione dei valori per il 2019
single_nr_2019 <- getValues(img_2019)
# Classificazione in due cluster
k_cluster_2019 <- kmeans(single_nr_2019, centers = 2)
# Creazione immagine classificata
img_2019_class <- setValues(img_2019[[1]], k_cluster_2019$cluster)
# Visualizzazione
plot(img_2019_class, col = c("blue", "yellow"), main = "Classificazione 2019")
\end{lstlisting}
Simile per 2020 e 2025.
\end{frame}

\begin{frame}{\textbf{Immagini Classificazione Land Cover}}
\begin{columns}
    \column{0.33\textwidth}
    \centering
    \textbf{Classificazione 2019}
    \includegraphics[width=\textwidth]{class_2019.png}
    
    \column{0.33\textwidth}
    \centering
    \textbf{Classificazione 2020}
    \includegraphics[width=\textwidth]{class_2020.png}
    
    \column{0.33\textwidth}
    \centering
    \textbf{Classificazione 2025}
    \includegraphics[width=\textwidth]{class_2025.png}
\end{columns}
\end{frame}

\section{Confronto Copertura Vegetale}

\begin{frame}[fragile]{\textbf{Confronto Copertura Vegetale}}
\textbf{Percentuali di copertura vegetale:}
\begin{lstlisting}
copertura_vegetale <- c("ridotta/assente", "buona")
# Supponendo che P_2019, P_2020, P_2025 siano definiti
Land_cover_perc <- data.frame(copertura_vegetale, P_2019, P_2020, P_2025)
\end{lstlisting}

\begin{columns}
    \column{0.33\textwidth}
    \centering
    \textbf{Copertura 2019}
    \includegraphics[width=\textwidth]{Cop_2019.png}
    
    \column{0.33\textwidth}
    \centering
    \textbf{Copertura 2020}
    \includegraphics[width=\textwidth]{Cop_2020.png}
    
    \column{0.33\textwidth}
    \centering
    \textbf{copertura 2025}
    \includegraphics[width=\textwidth]{Cop_2025.png}
\end{columns}
\end{frame}
\end{lstlisting}
\end{frame}

\section{Conclusioni}

\begin{frame}{\textbf{Conclusioni}}
\begin{itemize}
    \item Gli incendi del 2019 hanno ridotto significativamente la copertura vegetale.
    \item Lieve recupero osservato nel 2025, ma lontano dai livelli del 2019.
\end{itemize}
\end{frame}

\section{Studi Futuri}

\begin{frame}{\textbf{Studi Futuri}}
\begin{itemize}
    \item Aggiunta di ulteriori anni per un'analisi più dettagliata.
    \item Studio della biodiversità e correlazioni con i dati satellitari.
    \item Analisi dell'efficacia degli interventi di ripristino.
\end{itemize}
\end{frame}

\begin{frame}{}
\centering
\textbf{Grazie per l'attenzione!}
\end{frame}

\end{document}
